\documentclass[10pt,handout,english]{beamer}
\usetheme{Berlin}
\usecolortheme{seahorse}
\usepackage{amsmath}
\usepackage{listings}
\definecolor{mymauve}{rgb}{0.58,0,0.82}
\definecolor{mygreen}{rgb}{0,0.6,0}

\title[] % (optional, only for long titles)
{Python API for Mobile Robot Control}
\subtitle{Progress Presentation-1}
\author[AE-663 Course Project ] % (optional, for multiple authors)
{Parin Chheda (153076005) \\ Saurav Shandilya (153076004) \\ Group-10 }
\institute [Indian Institute of Technology Bombay]% (optional)
{
  
}
\date[\today] % (optional)
{Prof. Prabhu Ramchandra \\ PRof. Madhu Belur \\ Prof. Kumar Appaiah}
%\subject{Computer Science}

\begin{document} 
	\lstset{language=C,commentstyle=\color{mygreen},showspaces=true,   
		showstringspaces=false, basicstyle=\footnotesize,  
		breakatwhitespace=false,keepspaces=true,            breaklines=true,stepnumber=1,showspaces=false} 
\frame{\titlepage}

\begin{frame}{Objective}
\begin{enumerate}
	\item A Python API to control the different peripherals of the ${\mu}$controller
	\item Provide the user with a option of register level access of the ${\mu}$controller
	\item Allow the user to design an application without learning a new language and thoroughly knowing the architecture of the controller.
\end{enumerate}
\end{frame}

\begin{frame}{Block Diagram}

\end{frame}

\begin{frame}{Milestone Achieved}

\end{frame}

\begin{frame}{Future Work}

\end{frame}

\begin{frame}{User code snippet}
	\lstinputlisting[language=Python]{snippet.py}
\end{frame}
\end{document}